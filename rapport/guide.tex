\documentclass{pjsivi}


\title{Titre du projet}
\author{Nom des auteurs}
\date{2015-2016}


\begin{document}

\maketitleivi % affiche la page de garde (voir commandes \title et \author pour paramétrer)


\tableofcontents % affiche la table des matières


\chapter*{Introduction} % l'étoile * évite de donner un numéro de chapitre à l'introduction 
\addcontentsline{toc}{chapter}{Introduction} % même non numéroté, ce chapitre doit apparaitre dans la table des matières

Texte pour l'introduction


%--------------------------------------------------------------------------------------
%--------------------------------------------------------------------------------------
\chapter{Prise en main de \LaTeX} 


\section{Installation}

\subsection{Sous Linux}
Nombreuses documentations disponibles sur internet pour
l'installation des packages (voir package texlive; pour l'édition, geany suffit).

\subsection{Sous Windows}

\begin{itemize}
\item Pour compiler les fichiers tex, installer Miktex \cite{Mikex}
\item Pour écrire des documents Latex, installer TeXnicenter \cite{Texnicenter} (Freeware) ou WinEdt \cite{WinEdt} (Shareware).
\end{itemize}

\section{Compilation des documents}
Pour compiler un document \LaTeX en pdf :

\begin{itemize}
\item Faire \verb#pdflatex rapport#
\item Faire \verb#bibtex rapport# (génère la bibliographie)
\item Refaire \verb#pdflatex rapport# (intègre la bibliographie)
\item Vous pouvez bien sûr faire un \verb#Makefile# qui correspond ou utiliser l'éditeur s'il prévoit la compilation en pdf d'un document latex.
\end{itemize}

\section{Quelques commandes}
\subsection{Insertion de figures}

\begin{figure}[!ht]
  \centering
  \includegraphics[height=4cm]{logoLille1}
  \caption{Le logo de l'Université Lille 1.}
  \label{fig:ustl}
\end{figure}

Logo de l'USTL (voir fig. \ref{fig:ustl})

Mettre les images dans le répertoire \texttt{img}

\subsection{Insertion d'équations}

\begin{equation}
\int_{t=a}^{t=b}f(t)\;dt = 0 \label{eq:temps}
\end{equation}

D'après l'équation \ref{eq:temps} ... (voir par exemple \url{https://fr.wikibooks.org/wiki/LaTeX/Mathématiques} pour la syntaxe de l'écriture des formules).

\subsection{Insertion de code}

Pour plus d'infos sur le package listings, consulter cette note de bas de page\footnote{
\url{ftp://tug.ctan.org/pub/tex-archive/macros/latex/contrib/listings/listings.pdf}
}


\begin{lstlisting}[frame=trBL]
#include <iostream>

int main() {
    std::cout << "Hello, world!\n";
}
\end{lstlisting}

\subsection{Références et citations}

\begin{itemize}
\item Pour ajouter une référence : l'ajouter dans le fichier \verb#biblio.bib# (faire du copier coller avec les exemples qui s'y trouvent)
\item Pour citer une référence : exemple \verb#\cite{goossens93}# (référence présente dans \verb#biblio.bib#) pour obtenir \cite{goossens93}
\item {\bf Attention} : pour que les références et les citations apparaissent correctement dans le document, il faut exécuter \verb#bibtex# avant la compilation latex (sous geany : voir le menu \verb#build#)
\end{itemize}

\section{Doc \LaTeX}
Faire des recherches sur Google ou consulter ce livre très complet
\cite{goossens93}.

\chapter*{Conclusion} 
\addcontentsline{toc}{chapter}{Conclusion} % voir remarque de l'intro

Texte pour la conclusion

\bibliographystyle{unsrt} 
\bibliography{biblio} % affiche la bibliographie (voir voir fichier biblio.bib pour les entrées; seules les références citées par la commande \cite{ } apparaitront

\appendix  % on passe aux annexes : évite d'avoir le mot "Chapitre" qui apparait
\chapter*{Annexe 1}
\addcontentsline{toc}{chapter}{Annexe1} % voir remarque de l'intro

\end{document}
